\documentclass[NET,english]{tumbeamer}

% If you load additional packages, do so in packages.sty as figures are build
% as standalone documents and you may want to have effect on them, too.

% Folder structure:
% .
% ├── beamermods.sty                  % depricated an will be removed soon
% ├── compile                         % remotely compile slides
% ├── figures                         % all figures go here
% │   └── schichtenmodelle_osi.tikz   % each .tikz or .tex is a target
% ├── include                         % create your document here
% │   ├── example.tex                 % example document
% │   └── slides.tex                  % make document wide changes here
% ├── lit.bib                         % literature
% ├── Makefile
% ├── moeptikz.sty                    % fancy networking symbols
% ├── packages.sty                    % load additional packages there
% ├── pics                            % binary pcitures go here
% ├── slides.tex                      % main document (may be more than one)
% ├── tumbeamer.cls
% ├── tumcolor.sty                    % TUM color definitions
% ├── tumcontact.sty                  % TUM headers and footers
% ├── tumlang.sty                     % TUM names and language settings
% └── tumlogo.sty                     % TUM logos

% Configure author, title, etc. here:
\usepackage[utf8]{inputenc}
\usepackage{packages}
\usepackage{beamermods}

% For beamer mode (default):
\author[S. G\"unther]{Stephan Günther, M.\,Sc.}
\title[Networking]{Something with Networking}


% For lecture mode (use package option 'lecture'):
%\lecture[GRNVS]{Grundlagen Rechnernetze und Verteilte Systeme}
%\module{IN0010}
%\semester{SoSe\,2016}
%\assistants{Johannes Naab, Stephan Günther, Maurice Leclaire}


\usepackage{pgfpages}
\usepackage{ifthen}
% ============================================================================
% jobname solution
% ============================================================================
\newif\ifsolution%
\ifthenelse{\equal{\detokenize{notes}}{\jobname}}{%
\setbeameroption{show notes on second screen=bottom}
\setbeamercolor{note page}{bg=white, fg=black}
\setbeamercolor{note title}{bg=white!95!black, fg=black}
}{
}


\begin{document}

% For lecture mode, you may want to build one set of slides per chapter but
% with common page numbering. If so,
% 1) create a new .tex file for each chapter, e.g. slides_chapN.tex,
% 2) set the part counter to N-1 (assuming chapters start at 0), and
% 3) and name your chapter by using the \part{} command.
%\setcounter{part}{-1}
%\part{Organisatorisches und Einleitung}

% Include source files from ./include (or ./include/chapN).
\section{Section heading}

\begin{frame}
	\frametitle{Example frame}
	\begin{itemize}
		\item item 1
		\item $\ldots$
		\begin{itemize}
			\item test
			\item $\ldots$
		\end{itemize}
	\end{itemize}
	Citation \cite{rfc959}

	\paragraph{Math mode should be fully functional:}
	$$
	\hat s
	\overline s
	\mathcal S
	\mathbit S
	\mathbit \Lambda
	\sum
	\pd{\xi}
	\pr{X=0}
	\mathbit 1
	$$
\end{frame}

\begin{frame}
	\frametitle{Figures}
	\begin{figure}
		\centering
		\includegraphics[width=.5\textwidth]{figures/example}
		\caption{Figure caption}
		\label{Maizaso0}
	\end{figure}
	Figure~\ref{Maizaso0} shows a small network.
\end{frame}


% Include markdown source from ./pandoc
%\section{Section heading}

\begin{frame}
	\frametitle{Example frame}
	\begin{itemize}
		\item item 1
		\item $\ldots$
		\begin{itemize}
			\item test
			\item $\ldots$
		\end{itemize}
	\end{itemize}
	Citation \cite{rfc959}

	\paragraph{Math mode should be fully functional:}
	$$
	\hat s
	\overline s
	\mathcal S
	\mathbit S
	\mathbit \Lambda
	\sum
	\pd{\xi}
	\pr{X=0}
	\mathbit 1
	$$
\end{frame}

\begin{frame}
	\frametitle{Figures}
	\begin{figure}
		\centering
		\includegraphics[width=.5\textwidth]{figures/example}
		\caption{Figure caption}
		\label{Maizaso0}
	\end{figure}
	Figure~\ref{Maizaso0} shows a small network.
\end{frame}


% Comment out if you do not want a bibliography
\section{Bibliography}
\begin{frame}[allowframebreaks]
    \bibliographystyle{abbrv}
    \setbeamertemplate{bibliography item}[text]
    \footnotesize
    \bibliography{lit}
\end{frame}

\end{document}

