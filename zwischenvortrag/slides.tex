\documentclass[NET,english]{tumbeamer}

% If you load additional packages, do so in packages.sty as figures are build
% as standalone documents and you may want to have effect on them, too.

% Folder structure:
% .
% ├── beamermods.sty                  % depricated an will be removed soon
% ├── compile                         % remotely compile slides
% ├── figures                         % all figures go here
% │   └── schichtenmodelle_osi.tikz   % each .tikz or .tex is a target
% ├── include                         % create your document here
% │   ├── example.tex                 % example document
% │   └── slides.tex                  % make document wide changes here
% ├── lit.bib                         % literature
% ├── Makefile
% ├── moeptikz.sty                    % fancy networking symbols
% ├── packages.sty                    % load additional packages there
% ├── pics                            % binary pcitures go here
% ├── slides.tex                      % main document (may be more than one)
% ├── tumbeamer.cls
% ├── tumcolor.sty                    % TUM color definitions
% ├── tumcontact.sty                  % TUM headers and footers
% ├── tumlang.sty                     % TUM names and language settings
% └── tumlogo.sty                     % TUM logos

% Configure author, title, etc. here:
\usepackage[utf8]{inputenc}
\usepackage{packages}
\usepackage{beamermods}

% For beamer mode (default):
\author[S. G\"unther]{Stephan Günther, M.\,Sc.}
\title[Networking]{Something with Networking}


% For lecture mode (use package option 'lecture'):
%\lecture[GRNVS]{Grundlagen Rechnernetze und Verteilte Systeme}
%\module{IN0010}
%\semester{SoSe\,2016}
%\assistants{Johannes Naab, Stephan Günther, Maurice Leclaire}


\usepackage{pgfpages}
\usepackage{ifthen}
% ============================================================================
% jobname solution
% ============================================================================
\newif\ifsolution%
\ifthenelse{\equal{\detokenize{notes}}{\jobname}}{%
\setbeameroption{show notes on second screen=bottom}
\setbeamercolor{note page}{bg=white, fg=black}
\setbeamercolor{note title}{bg=white!95!black, fg=black}
}{
}


\begin{document}

% For lecture mode, you may want to build one set of slides per chapter but
% with common page numbering. If so,
% 1) create a new .tex file for each chapter, e.g. slides_chapN.tex,
% 2) set the part counter to N-1 (assuming chapters start at 0), and
% 3) and name your chapter by using the \part{} command.
%\setcounter{part}{-1}
%\part{Organisatorisches und Einleitung}

% Include source files from ./include (or ./include/chapN).
%\section{Section heading}

\begin{frame}
	\frametitle{Example frame}
	\begin{itemize}
		\item item 1
		\item $\ldots$
		\begin{itemize}
			\item test
			\item $\ldots$
		\end{itemize}
	\end{itemize}
	Citation \cite{rfc959}

	\paragraph{Math mode should be fully functional:}
	$$
	\hat s
	\overline s
	\mathcal S
	\mathbit S
	\mathbit \Lambda
	\sum
	\pd{\xi}
	\pr{X=0}
	\mathbit 1
	$$
\end{frame}

\begin{frame}
	\frametitle{Figures}
	\begin{figure}
		\centering
		\includegraphics[width=.5\textwidth]{figures/example}
		\caption{Figure caption}
		\label{Maizaso0}
	\end{figure}
	Figure~\ref{Maizaso0} shows a small network.
\end{frame}


% Include markdown source from ./pandoc
%\section{Section heading}

\begin{frame}
	\frametitle{Example frame}
	\begin{itemize}
		\item item 1
		\item $\ldots$
		\begin{itemize}
			\item test
			\item $\ldots$
		\end{itemize}
	\end{itemize}
	Citation \cite{rfc959}

	\paragraph{Math mode should be fully functional:}
	$$
	\hat s
	\overline s
	\mathcal S
	\mathbit S
	\mathbit \Lambda
	\sum
	\pd{\xi}
	\pr{X=0}
	\mathbit 1
	$$
\end{frame}

\begin{frame}
	\frametitle{Figures}
	\begin{figure}
		\centering
		\includegraphics[width=.5\textwidth]{figures/example}
		\caption{Figure caption}
		\label{Maizaso0}
	\end{figure}
	Figure~\ref{Maizaso0} shows a small network.
\end{frame}


\section{Why VPNs are Important}
\begin{frame}{Why VPNs are Important}
	\begin{itemize}
		\item Todays business is multi-national and international
		\item Many distributed sites that need to be interconnected
		\item Provide a secure channel for communication over insecure medium
		\item Other usecases: VM interconnects, cell tower backbones, firm intra-nets
	\end{itemize}
	
	=> Need for high throughput solutions
	
	%Current state-of-the-art: $\ll$10 Gbit/s~\footnote{64 KB packets, no packet loss, hardware accelerated ciphers only}
\end{frame}

\begin{frame}{Focus: Site-to-Site VPN}
	\begin{figure}
		\centering
		\includegraphics[width=1\linewidth]{figures/Site_to_Site_VPN}
		\caption{Overview of example Site-to-Site VPN setup}
		\label{fig:sitetositevpn}
	\end{figure}
	
	%Focus on Site-to-Site setups:
	\begin{itemize}
		\item Only two (or similar few) endpoints connecting many hosts
		\item Very high bandwidth between the gateways
	\end{itemize}
\end{frame}

\begin{frame}{Goals of this Thesis}
	\begin{itemize}
		\item Create benchmark criteria for Site-to-Site setups
		\item Evaluate performance of common implementations
		\item Develop a general performance model for VPNs
		\item Explore different approaches for performance improvements
	\end{itemize}
\end{frame}

\section{Overview of Common Implementations}
\begin{frame}{Overview of Common Implementations}
	\begin{columns}[T] % align columns
		\begin{column}{.30\textwidth}
			\textbf{OpenVPN}
			\begin{itemize}
				\item Pure Userspace\\
					Sockets
				\item TLS \& X.509
				\item L2 and L3
				\item Platform independent
			\end{itemize}
		\end{column}%
		\hfill%
		\begin{column}{.32\textwidth}
			\textbf{IPsec} (on Linux)
			\begin{itemize}
				\item Very complex\\
				protocol \& code
				\item Build into Kernel
				\item L3 only (without L2TP)
			\end{itemize}
		\end{column}%
		\hfill%
		\begin{column}{.35\textwidth}
			\textbf{WireGuard}
			\begin{itemize}
				\item Very new with\\ radical approaches
				\item State-of-the-Art cryptography
				\item Kernel module\\(inclusion ongoing)
				\item L3 only
			\end{itemize}
		\end{column}%
	\end{columns}

	\pause
	\vspace*{0.5em}
	\textbf{Shared problem}:
	
	Different degrees of slow under high load and in general
	
	No implementation achieves >5 Mpps or >10 Gbit/s\footnote{64 byte packets, COTS hardware}
	%TODO: comparison graph
\end{frame}

\begin{frame}{WireGuard under Load}

\end{frame}

\begin{frame}{Benchmarking: Traffic Shapes}
Exact traffic pattern and distributions vary depending on setup/use-case.
\textbf{Influences performance}.

Common metric is \textbf{number of flows}. Identifies a group of packets to a connection/subnet/host. Usually 3-tuple (L2 proto, L3 src, L3 dst) or 5-tuple (+ L4 ports).

\textbf{Packets-per-second} (Mpps) is more interesting than \textbf{bits-per-second} (Gbit/s). 64 byte packets (minimum Ethernet frame size), but more realistic distributions are possible.

\vspace*{-0.5em}
\begin{itemize}
	\item Single flow, high bandwidth
	
	Worst case, models single client-server setup
	
	\item Multiple flows, equal bandwidth
	
	Best case, fits site-to-site setups, easy to model
	
	\item "Elephant" flows (few number of flows dominate bandwidth-wise)
		
	Realistic case (Netflix, Youtube, ...)
\end{itemize}
\end{frame}

\begin{frame}{Example: Underutilization with single flows}
\begin{figure}
	%\centering
	%\hspace*{10em}
	\includegraphics[width=0.9\linewidth]{figures/wireguard-encrypt-64-single-flow}
	%\vspace*{-0.8em}
	\caption{WireGuard forwarding rate of 64 byte packets, single flow, X540-AT2}
	\label{fig:wireguard-encrypt-64-single-flow}
\end{figure}


\end{frame}

\begin{frame}
\begin{figure}
	%\centering
	%	\hspace*{-10em}
	\includegraphics[width=0.9\linewidth]{figures/queues_util_single_flow}

	\caption{CPU/Queues utilization under single flow traffic}
	\label{fig:queuesutilsingleflow}
\end{figure}

\vspace*{-1em}
\begin{itemize}
	\item NIC distributes packets to queues by L3 addresses
	\item One flow => everything in one queue
	\item Can be configured to include ports, not every NIC supports this
\end{itemize}
\end{frame}

\section{Improvements with MoonWire}
\begin{frame}{MoonWire}
	\begin{itemize}
		\item DPDK network stack to bypass slow kernel
		\item Lua for fast prototyping and interfacing with libraries (crypto)
		\item Aims for protocol compatibility with WireGuard
		\item Allows experimenting with different data structures, algorithms, ...
	\end{itemize}
\end{frame}

\begin{frame}{MoonWire Graphs}
	 %todo
	 [bytes/cycle graph of different ciphers]

	\begin{itemize}
		\item Cryptographic operations can become the bottleneck
		\item Hard to improve, correctness is more important
	\end{itemize}
	
	\textbf{Solution:} work distribution to multiple cores
	\begin{itemize}
		\item WireGuard utilizes Kernel worker tasks and a queue		
		\item Lots of possible implementations
	\end{itemize}
	
\end{frame}

\section{Case Study: Nonces in Symmetric Encryption}
\begin{frame}{Symmetric Encryption in a Nutshell}
	\texttt{encrypt(shared\_key, nonce, message) = ciphertext}
	

	\only<1>{
		\vspace*{1em}
		\begin{center}
			Should be easy to scale up?
		\end{center}
	}
	\pause
	
	\begin{itemize}
		\item Correct nonce generation/handling is critical. Nonce reuse (under the same key) breaks scheme and allows key recovery
		\item Nonce generation depends on length
			 
			 IETF ChaCha20 \& AES256 GCM: 96 bit (12 byte)
			 
			 Too short to be random (birthday problem)
			 
			 Recommendation: Counting up
			 
		\item Regular re-keying still recommended
	\end{itemize}
	
	But: Must be global over all threads/cores. Accessed for each packet => highly critical. Synchronization (mutex) and atomics are far too slow.
	
	%[Timing graphs for atomics/mutex vs. mpps]
\end{frame}

\begin{frame}{Nonce Generation Tricks}

	\begin{itemize}
		\item Partition nonce space per worker/CPU/thread: 
		
		8 bit worker\_id + 94 bit counter = 96 bit
		
		Worker$_{0}$: \textbf{0}123, \textbf{0}124, \textbf{0}125, ...
		
		Worker$_{1}$: \textbf{1}123, \textbf{1}124, \textbf{1}125, ...
		
		Beware of "overflows" into different worker partition
		
		\item Different cipher: XChaCha20 has 192 bit (24 byte) nonce
		
		Can be randomly generated safely
		
		Each worker has own PRNG instance (seeded carefully)
		
		Independent state \& no sharing => fast
		
		Trade-off: messages get larger (by 10 bytes), incompatible with existing protocol
	\end{itemize}

	\pause
	Good news: Decryption is much easier. Message contains everything.
\end{frame}

\begin{frame}{Multi-core Scaling Opportunities}
	%TODO
	Source IP \& port identical over all packets => Simple RSS does not work
	
	
\end{frame}

\begin{frame}{Remaining Work}
	\begin{itemize}
		\item More benchmarking \& measurements
		\item Try more other performance improvements
		\begin{itemize}
			\item AVX512 cipher implementations \& CPU downclocking
			\item NUMA
		\end{itemize}
		\item Thesis writing
	\end{itemize}
\end{frame}

% Comment out if you do not want a bibliography
%\section{Bibliography}
%\begin{frame}[allowframebreaks]
%    \bibliographystyle{abbrv}
%    \setbeamertemplate{bibliography item}[text]
%    \footnotesize
%    \bibliography{lit}
%\end{frame}

\end{document}

